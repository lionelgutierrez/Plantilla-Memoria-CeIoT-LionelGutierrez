% Chapter 1

\chapter{Introducción general} % Main chapter title

\label{Chapter1} % For referencing the chapter elsewhere, use \ref{Chapter1} 
\label{IntroGeneral}

Poner párrafo introductorio.

%----------------------------------------------------------------------------------------

% Define some commands to keep the formatting separated from the content 
\newcommand{\keyword}[1]{\textbf{#1}}
\newcommand{\tabhead}[1]{\textbf{#1}}
\newcommand{\code}[1]{\texttt{#1}}
\newcommand{\file}[1]{\texttt{\bfseries#1}}
\newcommand{\option}[1]{\texttt{\itshape#1}}
\newcommand{\grados}{$^{\circ}$}

%----------------------------------------------------------------------------------------

%\section{Introducción}

%----------------------------------------------------------------------------------------
\section{Estado del arte}

Introducción, propósito y estado del arte de IoT y solución propuesta.

\subsection{Tecnología IoT}

Introducción a las soluciones IoT, posibilidades que brinda para sensado y control de diferentes procesos, ventajas de la tecnología (economía, simplicidad).

\subsection{Control de acceso}

Sistemas de control de acceso que existen en el mercado. Diferencias con el sistema propuesto (valor agregado de la propuesta contra soluciones existentes).

%----------------------------------------------------------------------------------------

\section{Motivación}

Razones por el cual se desea desarrollar el sistema. Justificaciones. Necesidad del cliente.

%----------------------------------------------------------------------------------------

\section{Objetivos y alcance}

Objetivos y alcance del trabajo.

%----------------------------------------------------------------------------------------

