% Chapter Template

\chapter{Conclusiones} % Main chapter title

\label{Chapter5} % Change X to a consecutive number; for referencing this chapter elsewhere, use \ref{ChapterX}


%----------------------------------------------------------------------------------------

%----------------------------------------------------------------------------------------
%	SECTION 1
%----------------------------------------------------------------------------------------

\section{Resultados obtenidos}

Se logró cumplir el alcance y objetivo del proyecto. En primer lugar, se implementó el sistema de control solicitado que incluyó la puesta en funcionamiento de un módulo sensor, un módulo actuador y una aplicación web para la gestión de tareas de control y de alertas. En segundo lugar, se sentaron las bases para el agregado de futuros casos de uso, para lo cual se hizo un diseño modular. Esto permitirá al sistema el sensado de datos de nuevos procesos de planta, según el requerimiento 1.1, especificado en la sección \ref{sec:Requerimientos}.

Adicionalmente, se incorporó al sistema un módulo para gestionar la autenticación y autorización de los usuarios y el acceso a la API Rest del backend del sistema de modo seguro. Esto posibilitó cumplir con el requerimiento 1.6, que fue agregado al trabajo durante su desarrollo. Este requerimiento permite independizar al sistema desarrollado del sistema de autenticación de la empresa y lograr una mayor portabilidad, lo que le da la posibilidad a futuro de expandir el mismo a otras empresas.

Finalmente, el trabajo cumplió con el requerimiento de lograr una gestión efectiva de los terceros, al implementar un proceso de control estricto de los requisitos de ingreso. Si bien existen otros sistemas de control de ingreso en el mercado, se logró agregar valor mediante la comunicación del sistema en cuestión con el sistema de control de documentación de terceros sumado a los procesos de alerta y gestión implementados. El trabajo realizado habilita una gestión proactiva, rápida y ordenada de los terceros, de forma de actuar inmediatamente ante problemas de ingresos. Las ventajas obtenidas incluyen:

\begin{itemize}
\item Reducir tiempo para la gestión.
\item Evitar atrasos en ingresos por falta de ajustes en la documentación.
\item Evitar problemas legales ante incidentes del personal externo.
\item Evitar el uso de papel y herramientas des-centralizadas para el control y gestión de los terceros por parte de cada sector de la empresa. 
\end{itemize}

Si bien todavía el sistema no se implementó en operativo, con las pruebas realizadas y analizando los ingresos de personal en los últimos dos años, se prevé evitar un 5\% de ingresos incorrectos o con problemas, y reducir los tiempos ante inconvenientes con la documentación en unas 10 horas hombre/mes.

Cabe destacar la importancia de los conocimientos obtenidos a lo largo de la carrera. En primer lugar, fueron muy importantes los aportes de la asignatura de gestión de proyectos. Una buena gestión de proyectos fue fundamental, tanto para lograr una planificación clara que actúe como guía a lo largo de todo el desarrollo, como para minimizar riesgos. En lo particular del trabajo, luego del comienzo del desarrollo se solicitó agregar un nuevo requerimiento al mismo. Dado que se contaba con un diagrama de Gantt \citep{WEBSITE:DiagGantt} detallado y el avance real al momento de la solicitud, se pudo determinar que el nuevo requisito podía cumplirse en tiempo y forma, sin penalizar la fecha de finalización del proyecto. Sin una gestión de proyectos clara, es muy probable que este nuevo requerimiento haya sido rechazado.

Adicionalmente, los conocimientos en desarrollo de aplicaciones multiplataforma permitieron plantear una solución que sea \textit{web responsive} y que a futuro se puede implementar en un entorno mobile con mínimo esfuerzo. También se abre la posibilidad de implementar el sistema en la nube.


%----------------------------------------------------------------------------------------
%	SECTION 2
%----------------------------------------------------------------------------------------
\section{Trabajo futuro}

Para la continuidad y mejora de este trabajo se plantean dos líneas de acción.

Como primera línea de acción, referida al trabajo desarrollado, se incluyen:

\begin{itemize}
\item Realizar mejoras en seguridad:

   \begin{itemize}
      \item Agregar comunicación HTTPS entre los diferentes módulos del sistema. De este modo, se asegura que la información viaje encriptada y se evitan ataques del tipo \textit{man in the middle}. Esto permitirá migrar el sistema a la nube, donde las comunicaciones viajan a través de Internet y no son seguras.
      \item Agregar un segundo factor de autenticación al sistema. Con el objetivo de evitar que ante pérdidas o robos de la tarjeta RFID de ingreso o duplicación de la misma un atacante pueda ingresar a planta, se plantea la posibilidad de agregar un teclado matricial de forma de requerir además de la tarjeta una clave numérica durante el proceso de ingreso.
   \end{itemize}
      
   \item Automatizar tareas de configuración: se analiza implementar una aplicación para poder configurar las acciones del sistema ante las diferentes variantes de entradas (ingreso correcto, ingreso de usuario inactivo, ingreso con documentación vencida). Actualmente esta información se guarda y administra en una base de datos no relacional, por parte del personal de sistemas, que permite definir para cada tipo y valor de entrada un conjunto de acciones de salidas (tareas de control, alertas, emails). Se evalúa desarrollar una aplicación para que el usuario pueda configurar estas salidas y generar diferentes tipos de acción, independizándose del área de sistemas.
   \item Realizar una prueba de implementación en la nube: utilizar la nube de Azure para probar y asegurar el escalamiento de la solución. Esto permitirá incluir nuevas locaciones o plantas industriales al trabajo, ya sea dentro de la empresa actual o para ser implementado en nuevas empresas.
\end{itemize}   

Como segunda línea de acción, en el marco del proyecto integral de gestión de alertas y procesos, el objetivo es incorporar nuevos procesos y casos de uso al sistema. De hecho, ya fue solicitado un primer caso de uso por parte del laboratorio de metrología de la empresa. El mismo implica el control de temperatura y humedad de dicho laboratorio, para asegurar que ambas variables se encuentren dentro de los límites requeridos y generar alertas en caso de desvíos para poder actuar en consecuencia, manual o automáticamente. 
