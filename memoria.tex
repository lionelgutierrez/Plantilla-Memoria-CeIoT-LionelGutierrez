%%%%%%%%%%%%%%%%%%%%%%%%%%%%%%%%%%%%%%%%%%%%%%%%%%%%%%%%%%%%%%%%%%%%%%%%%%%%%%%%
%
% Template license:
% CC BY-NC-SA 3.0 (http://creativecommons.org/licenses/by-nc-sa/3.0/)
%
%%%%%%%%%%%%%%%%%%%%%%%%%%%%%%%%%%%%%%%%%%%%%%%%%%%%%%%%%%%%%%%%%%%%%%%%%%%%%%%%

%----------------------------------------------------------------------------------------
%	PACKAGES AND OTHER DOCUMENT CONFIGURATIONS
%----------------------------------------------------------------------------------------

\documentclass[
11pt, % The default document font size, options: 10pt, 11pt, 12pt
%oneside, % Two side (alternating margins) for binding by default, uncomment to switch to one side
%chapterinoneline,% Have the chapter title next to the number in one single line
spanish,
singlespacing, % Single line spacing, alternatives: onehalfspacing or doublespacing
%draft, % Uncomment to enable draft mode (no pictures, no links, overfull hboxes indicated)
%nolistspacing, % If the document is onehalfspacing or doublespacing, uncomment this to set spacing in lists to single
%liststotoc, % Uncomment to add the list of figures/tables/etc to the table of contents
%toctotoc, % Uncomment to add the main table of contents to the table of contents
parskip, % Uncomment to add space between paragraphs
%codirector, % Uncomment to add a codirector to the title page
headsepline, % Uncomment to get a line under the header
]{MastersDoctoralThesis} % The class file specifying the document structure
\usepackage{multirow}
\usepackage{enumitem}
\usepackage{longtable}

%----------------------------------------------------------------------------------------
%	INFORMACIÓN DE LA MEMORIA
%----------------------------------------------------------------------------------------

\thesistitle{Sistema de gestión de alertas y tareas de procesos de planta - control de acceso} % El títulos de la memoria, se usa en la carátula y se puede usar el cualquier lugar del documento con el comando \ttitle

% Nombre del posgrado, se usa en la carátula y se puede usar el cualquier lugar del documento con el comando \degreename
%\posgrado{Carrera de Especialización en Sistemas Embebidos} 
\posgrado{Carrera de Especialización en Internet de las Cosas} 
%\posgrado{Carrera de Especialización en Intelegencia Artificial}
%\posgrado{Maestría en Sistemas Embebidos} 
%\posgrado{Maestría en Internet de las cosas}

\author{Ing. Lionel Gutierrez} % Tu nombre, se usa en la carátula y se puede usar el cualquier lugar del documento con el comando \authorname

\director{Ing. Gustavo Ramoscelli (UNS)} % El nombre del director, se usa en la carátula y se puede usar el cualquier lugar del documento con el comando \dirname
\codirector{Nombre del codirector (pertenencia)} % El nombre del codirector si lo hubiera, se usa en la carátula y se puede usar el cualquier lugar del documento con el comando \codirname.  Para activar este campo se debe descomentar la opción "codirector" en el comando \documentclass, línea 23.

\juradoUNO{Ing. José Alamos (HAW Hamburg)} % Nombre y pertenencia del un jurado se usa en la carátula y se puede usar el cualquier lugar del documento con el comando \jur1name
\juradoDOS{Mg. Ing. Leandro Lanzieri Rodriguez (UTN FRA/HAW Hamburg)} % Nombre y pertenencia del un jurado se usa en la carátula y se puede usar el cualquier lugar del documento con el comando \jur2name
\juradoTRES{Esp. Lic. Leopoldo Zimperz (UBA)} % Nombre y pertenencia del un jurado se usa en la carátula y se puede usar el cualquier lugar del documento con el comando \jur3name

%\ciudad{Ciudad Autónoma de Buenos Aires}
\ciudad{ciudad de Villa Mercedes}

\fechaINICIO{mayo de 2020}
\fechaFINAL{marzo de 2021}


\keywords{Sistemas embebidos, FIUBA} % Keywords for your thesis, print it elsewhere with \keywordnames


\begin{document}


\frontmatter % Use roman page numbering style (i, ii, iii, iv...) for the pre-content pages

\pagestyle{plain} % Default to the plain heading style until the thesis style is called for the body content


%----------------------------------------------------------------------------------------
%	RESUMEN - ABSTRACT 
%----------------------------------------------------------------------------------------

\begin{abstract}
\addchaptertocentry{\abstractname} % Add the abstract to the table of contents
%
%The Thesis Abstract is written here (and usually kept to just this page). The page is kept centered vertically so can expand into the blank space above the title too\ldots
\centering

La presente memoria describe el diseño e implementación de un sistema de control de acceso de personal de terceros a una locación industrial. El sistema garantiza que solo aquellas personas que tienen en regla los requisitos legales y médicos solicitados accedan. De esta forma se evita que la empresa sea responsable ante posibles accidentes o incidentes de dicho personal. El trabajo desarrollado es la primera etapa de un proyecto integral de gestión de alertas y procesos para la empresa Tenaris Metalmecánica, sobre el cual se agregarán a futuro nuevos casos de uso.

Para la elaboración del trabajo se aplicaron conocimientos adquiridos a lo largo de la carrera, principalmente los referidos a gestión de proyectos, desarrollo de aplicaciones web y multiplataforma, protocolos de Internet y seguridad en IoT. Además, se integraron tecnologías de base de datos relacionales y no relacionales y se aplicaron varias técnicas de testing.

\end{abstract}

%----------------------------------------------------------------------------------------
%	CONTENIDO DE LA MEMORIA  - AGRADECIMIENTOS
%----------------------------------------------------------------------------------------

\begin{acknowledgements}
%\addchaptertocentry{\acknowledgementname} % Descomentando esta línea se puede agregar los agradecimientos al índice
\vspace{1.5cm}

Especialmente a mi compañera Belén, que me ha acompañado y apoyado a lo largo del cursado de la Especialización.

A los profesores y compañeros de la Carrera de Especialización por compartir sus conocimientos y estar siempre a disposición para ayudar.

\end{acknowledgements}

%----------------------------------------------------------------------------------------
%	LISTA DE CONTENIDOS/FIGURAS/TABLAS
%----------------------------------------------------------------------------------------

\tableofcontents % Prints the main table of contents

\listoffigures % Prints the list of figures

\listoftables % Prints the list of tables


%----------------------------------------------------------------------------------------
%	CONTENIDO DE LA MEMORIA  - DEDICATORIA
%----------------------------------------------------------------------------------------

%\dedicatory{\textbf{Dedicado a... [OPCIONAL]}}  % escribir acá si se desea una dedicatoria

%----------------------------------------------------------------------------------------
%	CONTENIDO DE LA MEMORIA  - CAPÍTULOS
%----------------------------------------------------------------------------------------

\mainmatter % Begin numeric (1,2,3...) page numbering

\pagestyle{thesis} % Return the page headers back to the "thesis" style

% Incluir los capítulos como archivos separados desde la carpeta Chapters

% Chapter 1

\chapter{Introducción general} % Main chapter title

\label{Chapter1} % For referencing the chapter elsewhere, use \ref{Chapter1} 
\label{IntroGeneral}

En el presente capítulo se presentan los diferentes métodos de control de acceso en el ámbito de Internet de las Cosas y se expone la motivación que impulsa este trabajo, los objetivos y el alcance definidos.

%----------------------------------------------------------------------------------------

% Define some commands to keep the formatting separated from the content 
\newcommand{\keyword}[1]{\textbf{#1}}
\newcommand{\tabhead}[1]{\textbf{#1}}
\newcommand{\code}[1]{\texttt{#1}}
\newcommand{\file}[1]{\texttt{\bfseries#1}}
\newcommand{\option}[1]{\texttt{\itshape#1}}
\newcommand{\grados}{$^{\circ}$}

%----------------------------------------------------------------------------------------

%\section{Introducción}

%----------------------------------------------------------------------------------------
\section{Estado del arte}

Es esta sección se presenta una introducción a las soluciones IoT y su uso para gestionar el control de acceso en las empresas.

\subsection{Tecnología IoT}

Con la expansión de Internet y las tecnologías de conectividad móvil 3G, 4G y la nueva tecnología 5G, se ha producido una revolución en el acceso a la información que tiene un impacto sobre la educación, el modo de comunicarnos, las empresas, la ciencia, el gobierno y la humanidad en general. En este contexto Internet de las Cosas representa la próxima revolución de Internet, dado que permitirá dar un gran salto en la capacidad de reunir, analizar y distribuir datos, los que que podemos convertir en información, conocimiento, y en última instancia, sabiduría \citep{WEBSITE:IOT}.

Internet de las Cosas (IoT) es un concepto que se refiere a la interconexión digital de objetos cotidianos con Internet \citep{WEBSITE:IOTDefinicion}. El concepto de Internet de las Cosas fue propuesto en 1999, por Kevin Ashton, en el AutoID Center del MIT, en donde se realizaban investigaciones sobre RFID y tecnologías de sensores \citep{WEBSITE:IOTMIT}. Según el Grupo de soluciones empresariales basadas en Internet (IBSG, Internet Business Solutions Group) de Cisco, IoT es sencillamente el punto en el tiempo en el que se conectaron a Internet más ``cosas u objetos'' que personas. En 2003, había aproximadamente 6,3 mil millones de personas en el planeta, y había 500 millones de dispositivos conectados a Internet. Si dividimos la cantidad de dispositivos conectados por la población mundial, el resultado indica que había menos de un dispositivo (0,08) por persona. Sin embargo, el crecimiento explosivo de los teléfonos inteligentes y las tablets luego de esos años elevó a 12,5 mil millones en 2010 la cantidad de dispositivos conectados a Internet, en tanto que la población mundial aumentó a 6,8 mil millones, por lo que el número de dispositivos conectados por persona pasó a ser superior a 1 (1,84 para ser exactos) por primera vez en la historia. Al mismo tiempo, se preve que para el 2025 tendremos alrededor de 41.600 millones de dispositivos conectados \citep{WEBSITE:IOTFechas}.

IoT se trata principalmente de una red de interconexión digital entre objetos, personas e Internet, que permite el intercambio de datos con otros dispositivos. Esto hace que se pueda capturar información clave sobre el uso y el rendimiento de objetos para así detectar patrones, hacer recomendaciones, mejorar la eficiencia y crear experiencias únicas para los usuarios. Esta tecnología está transformando la vida de las personas y las empresas, impulsando innovaciones nunca antes vistas. La misma tiene el potencial de mejorar drásticamente la manera en que las personas viven, aprenden, trabajan y se entretienen. Algunos ejemplos son impactantes, por ejemplo, los pacientes ingieren dispositivos de Internet que ingresan a su cuerpo para ayudar a los médicos a diagnosticar y determinar las causas de ciertas enfermedades y es posible colocar sensores pequeñísimos en plantas, animales y fenómenos geológicos y conectarlos a Internet para medirlos, estudiarlos y prever como se comportarán a futuro. En la figura \ref{fig:Iot} se muestra un diagrama donde se ven las interconexiones de IoT entre dispositivos y los elementos asociados.

\begin{figure}[ht]
	\centering
	\includegraphics[width=1\textwidth]{./Figures/iot.png}
	\caption{Interconexion de dispositivos y tecnologías en IoT.}
	\label{fig:Iot}
\end{figure}

Resulta importante destacar que existe una correlación directa entre los datos y la sabiduría. Cuántos más datos se generan, más conocimiento y sabiduría pueden obtener las personas. IoT aumenta drásticamente la cantidad de datos que están disponibles para que los procesemos. Este aumento, combinado con la capacidad de Internet de comunicar estos datos, hará posible que las personas avancen aún más.En la figura \ref{fig:Sabiduria} se muestra un diagrama de la correlación entre datos y sabiduría.

\begin{figure}[ht]
	\centering
	\includegraphics[width=1\textwidth]{./Figures/sabiduria.png}
	\caption{Correlación entre datos y sabiduría.}
	\label{fig:Sabiduria}
\end{figure}

Si se combina la capacidad de la próxima evolución de Internet (IoT) para percibir, recolectar, transmitir, analizar y distribuir datos a escala masiva con la manera en que las personas procesan la información, la humanidad tendrá el conocimiento y la sabiduría necesarios no solo para sobrevivir sino para mejorar y prosperar en los próximos meses, años, décadas y siglos.

En lo particular de las empresas, con IoT aparece el término de transformación digital. La misma es la aplicación de la tecnología digital para proporcionar el entorno adecuado para la innovación de las empresas y la industria. En este contexto, el desarrollo de nuestro trabajo propone realizar una transformación digital en la empresa Tenaris Metalmecánica, aprovechando estas tecnologías disruptivas.

\subsection{Control de acceso}

El control de acceso se refiere a los mecanismos que permiten o restringen la entrada de una persona o vehículo a una empresa o recinto mediante su identificación. Dentro de los principales objetivos del control de acceso incluye garantizar la seguridad y facilitar la organización empresarial. Cuando una organización instala un sistema de control de acceso, lo hace básicamente pensando en tres propósitos:
\begin{itemize}
\item Cuidar de la integridad física de las personas; es decir, evitar que ataquen a alguien.
\item Proteger la información de la compañía: bases de datos, material sensible, etc.
\item Custodiar los activos de la empresa, como los equipos electrónicos o cualquier otro bien que sea vendible.
\end{itemize}

e emplean diferentes medios para monitorear y controlar el acceso de las personas a una instalación. Décadas atrás se usaban sistemas de cerraduras y llaves; sin embargo, además de ser vulnerables, las llaves robadas o extraviadas representaban gastos adicionales para las empresas. Con el advenimiento de Internet, y principalmente de IoT el control de acceso migró a sistemas más robustos con credenciales electrónicas o identificación biométrica.

\subsubsection{Factores de autenticación}
Para un proceso de identificación, sea físico o digital, se comprueba la identidad de la persona que hace la solicitud. Esa verificación se puede ejecutar usando uno o varios factores de autenticación.

Los factores de autenticación se pueden dividir en:

\begin{itemize}
\item Lo que sé: el conocimiento que la persona tiene, puede ser un PIN, una contraseña o un patrón.
\item Lo que tengo: la identificación que posee un individuo para certificar que es él, como una credencial física o virtual.
\item Lo que soy: los rasgos corporales únicos de la persona que se utilizan para verificar la identidad (biometría).
\end{itemize}

Para aumentar el nivel de seguridad, los sistemas modernos implementan varios factores de autenticación en los puntos de acceso, combinando ``lo que tengo'' con ``lo que sé'' y con ``lo que soy'' \citep{WEBSITE:ControlAcceso}.

\subsubsection{Clasificación de sistemas de control de acceso}

Los sistemas de control de acceso para personas se clasifican por dos criterios: conexiones y método de identificación \citep{WEBSITE:ControlAccesoPersonas}.

Por su conectividad:

\begin{itemize}
\item Controles de acceso autónomos: No necesitan conectarse a la red y no guardan datos de los movimientos que se produzcan, sino que se limitan a abrir las puertas o barreras. 
\item Controles de acceso conectados en red: Estos, además de abrir accesos, registran las entradas y salidas de personas. Deben conectarse a Internet ya que la información de esos movimientos se descargará en una aplicación para poder generar informes.
\end{itemize}

Por su método de identificación:

\begin{itemize}
\item Biométricos: La identificación se produce mediante la lectura de datos físicos individuales que imposibilitan la suplantación al ser intransferibles, por lo que se consideran los más seguros. Su empleo implica el cumplimiento de la normativa en materia de protección de datos y no está permitida en todas las. Dentro de estas tenemos reconocimiento facial y huella dactilar.
\item Tarjetas: En muchas oficinas, y también en centros de trabajo como laboratorios, talleres o fábricas donde los trabajos manuales y la higiene no aconsejen utilizar la huella dactilar, se utilizan llaveros y tarjetas para identificarse en los lectores. Estas últimas son de dos clases:
	\begin{itemize}
	\item Tarjetas magnéticas: Se deben introducir en el lector por el lado de la banda magnética que contiene los datos de cada persona autorizada para abrir el acceso.
	\item Tarjetas (RFID): No requieren contacto con el lector para activar el relé que abre el acceso por radiofrecuencia, y por eso se llaman ``tarjetas de proximidad''.
	\end{itemize}
\item Contraseña numérica: algunos sistemas de control de accesos permiten fichar poniendo una contraseña en el teclado del propio terminal.
\end{itemize}

\subsubsection{Estudio de mercado}

Para nuestro trabajo se realizó un análisis de los sistemas existentes en el mercado y se encontró que la seguridad y confiabilidad de las soluciones existentes van en relación a su precio. Se detectó que la gran mayoría de los sistemas del mercado son cerrados y auto-gestionados, lo que limita su integración con otros sistemas y el valor agregado que podría darse al trabajo a implementar. Existen productos que pueden integrarse con asistentes de voz, como Echo o Alexa de Amazon o Home de Google. El problema es que a nivel empresarial la privacidad de los datos es fundamental, lo que implica que no pueden o no es deseable compartir datos con los mismos. Varios de los sistemas permiten acceso mediante huella, lo cual puede resultar muy cómodo, pero no es útil en la locación industrial donde se va a implementar el trabajo debido a que se requiere por un lado una política especial para el guardado de los datos y no divulgación de las huellas dactilares y por otro lado por el tipo de trabajo manual y falta de higiene se desaconseja. En el capítulo 4, en la sección \ref{sec:comparativa} se realiza un análisis comparativo de las soluciones estudiadas y las ventajas y desventajas de ellas contra el trabajo desarrollado. En particular se analizaron dos soluciones, una de Pronext (Pronext KY800 \citep{WEBSITE:Ponext}) y una de Samsung (Samsung SHS-H505 \citep{WEBSITE:Samsung}). Si bien dichas soluciones no son costosas y brindan algunas características interesantes como doble factor de autenticación o acceso mediante huella, no soportan conexión con sistemas externos lo cual no permite el agregado de valor a los datos, para transformarlos en información para la toma de decisiones. 


%----------------------------------------------------------------------------------------

\section{Motivación}

La empresa Tenaris Metalmecánica produce varillas de bombeo para la extracción de petróleo, para lo cual tiene una planta industrial con diferentes equipos y procesos para llevar adelante la fabricación de estos productos. Ante la necesidad continua de mejorar la calidad de los procesos de producción y los productos manufacturados a clientes, se requiere contar con alertas tempranas ante desvíos en los procesos industriales y de soporte en la empresa. Se ha detectado que existen muchos problemas recurrentes en los procesos de planta, los cuales se plantean en reuniones diarias de gestión, pero que quedan sin solución y vuelven a repetirse por no abordarlos de una manera sistemática. Muchos de estos problemas implican soluciones rápidas y automáticas y en otros casos implican generar tareas que requieren tiempo e involucramiento de diferentes actores o sectores de planta. Estos problemas han derivado a lo largo del tiempo en problemas de calidad (no conformidades), pérdida de dinero y tiempo de recursos valiosos para la empresa. 

En este sentido, se plantea la necesidad de contar con un sistema que permita detectar rápidamente los desvíos en los procesos y generar alertas tempranas o actuar automáticamente en la medida de lo posible. Existen casos donde los procesos a controlar tienen cierto grado de automatismo y se puede generar respuestas automáticas, mediante diferentes tipos de actuadores. En otros casos se generan alertas para que el personal o la gerencia tome decisiones y hay otros casos donde se deben generar tareas de control o mejoras a implementar por parte de diferentes sectores. De este modo, se logra una trazabilidad entre el problema o desvío y las acciones correctivas o preventivas a futuro. Para el caso de las tareas o flujos de trabajo manuales, el sistema debe permitir un seguimiento de las mismas y se brindar información de las tareas completas, en curso y la antigüedad de las mismas.

El sistema debe permitir la generación de alertas y la actuación desde distintos puntos, procesos y tecnologías, mediante un sistema que expone una interfaz de servicios web para su comunicación. Además, debe adaptarse a diferentes casos de uso o procesos que se irán agregando en diferentes etapas.

%----------------------------------------------------------------------------------------

\section{Objetivos y alcance}

En esta primera etapa y para el trabajo proyectado el objetivo será controlar los ingresos de terceros a la planta, para asegurar que todo tercero que acceda a planta cuenta con todos los requisitos legales y médicos al día. En caso contrario, se debe inhabilitar su acceso e informar a las áreas operativas. Se toma este caso de uso como primera prioridad, dado que cualquier accidente o problema que pueda suceder con un tercero dentro de la planta, si el mismo no contara con toda la documentación necesaria, podría incurrirse en graves problemas legales para la empresa. Durante el último año se detectaron en auditorías internas varios casos de usuarios con documentación vencida y se necesita actuar en consecuencia en lo inmediato.

El alcance de este proyecto incluye el desarrollo de una plataforma de software y módulos de actuación y sensado para el control de ingreso de terceros a planta. A su vez, la plataforma deberá quedar preparada para la incorporación futura de nuevos módulos de sensado y actuación, que deberán ser desarrollados y luego configurados pertinentemente en la plataforma, pero sin necesidades de mayores desarrollos en la misma.

Dentro de la plataforma de software se implementarán los servicios necesarios para la recepción de alarmas mediante servicios web, se generará la infraestructura y modelos de datos necesarios para modelizar alertas, tareas y actuadores genéricos. Adicionalmente, se desarrollará un módulo de sensado para leer tarjetas RFID, las cuales se asignarán a cada usuario tercero que quiera ingresar a planta y un módulo de actuación para liberar o bloquear la cerradura de entrada a la planta junta a una alarma visual que indique la habilitación o no del usuario. Ambos módulos serán componentes electrónicos o controladores que se encargarán del sensado o actuación y la comunicación de estos sensores y actuadores con la plataforma de software, bien sea mediante una red cableada o Wifi, según la disponibilidad de infraestructura del área. La plataforma de software se implementará dentro de la Intranet de la empresa, en la arquitectura existente.

No se implementarán los módulos para la configuración automática por parte del usuario de las tareas, alertas y actuadores, de modo que estas configuraciones se harán cargando datos directamente en la base de datos del sistema. Queda para una etapa posterior del proyecto desarrollar este módulo, para facilitar la configuración y agregado de nuevos módulos al sistema por parte de los usuarios, sin depender del área de sistemas. Tampoco se desarrollarán otros módulos adicionales al módulo de actuación y sensado para el control de terceros.

Para determinar si un usuario está habilitado o no a fín de ingresa a planta se consultará con un sistema de documentación de terceros que ya está operativo en la empresa. El mismo permite conocer si el tercero está activo (está prestando servicios actualmente en la empresa o fue dado de baja por fin de su contratación) y si tiene toda la documentación requerida al día. Nuestro desarrollo se comunicará con este a través de la Intranet de la empresa.

En la figura \ref{fig:Solucionbasica} se muestra el diagrama en bloques de la solución, con las interfaces del sistema: el ingreso de información desde el módulo de sensado, la interfaz de conexión entre el sistema de alertas y el sistema de documentación de terceros y las salidas del sistema al módulo de actuación, las alertas por email y las tareas generadas por el sistema a los usuarios.

\begin{figure}[ht]
	\centering
	\includegraphics[width=1\textwidth]{./Figures/solucionbasica.png}
	\caption{Diagrama en bloques de la solución propuesta.}
	\label{fig:Solucionbasica}
\end{figure}




%----------------------------------------------------------------------------------------


\chapter{Introducción específica} % Main chapter title

\label{Chapter2}

%----------------------------------------------------------------------------------------
%	SECTION 1
%----------------------------------------------------------------------------------------

Poner párrafo introductorio.

\section{Protocolos de comunicación}

Descripción de los protocolo de comunicación (Wi-Fi/HTTP) utilizados para IoT.

\subsection{Tecnología de comunicación Wi-Fi}

Descripción de tecnología Wi-Fi.

\subsection{Protoclo HTTP}

Descripción de protocolo HTTP.

\section{Componentes de Hardware utilizado}

Descripción de los componentes de hardware utilizados: ESP32, lector de tarjetas, cerradura electrónica.

\subsection{Módulo ESP32}

Descripción del módulo.

\subsection{Módulo RFID RC522}

Descripción del módulo.

\subsection{Cerradura electrónica}

Descripción de cerradura.

\section{Tecnologías de Software aplicadas}
 
Descripción de las tecnologías de software utilizadas.

\subsection{Node.JS}
\subsection{Ionic}
\subsection{PostgreSQL}
\subsection{MongoDB}
\subsection{Docker}
\subsection{Postman}


\section{Software de control de versiones}

Descripción del software de control de versiones.

\subsection{GitFlow}

Descripción de la herramienta.

\section{Requerimientos}
 
Requerimientos del proyecto, tanto funcionales, no funcionales, de documentación y de validación. Enumeración de los mismos.
 
\subsection{Requerimientos funcionales}
\subsection{Requerimientos no funcionales}
\subsection{Requerimientos de documentación}
\subsection{Requerimientos de validación}

 
\chapter{Diseño e implementación} % Main chapter title

\label{Chapter3} % Change X to a consecutive number; for referencing this chapter elsewhere, use \ref{ChapterX}

\definecolor{mygreen}{rgb}{0,0.6,0}
\definecolor{mygray}{rgb}{0.5,0.5,0.5}
\definecolor{mymauve}{rgb}{0.58,0,0.82}

%%%%%%%%%%%%%%%%%%%%%%%%%%%%%%%%%%%%%%%%%%%%%%%%%%%%%%%%%%%%%%%%%%%%%%%%%%%%%
% parámetros para configurar el formato del código en los entornos lstlisting
%%%%%%%%%%%%%%%%%%%%%%%%%%%%%%%%%%%%%%%%%%%%%%%%%%%%%%%%%%%%%%%%%%%%%%%%%%%%%
\lstset{ %
  backgroundcolor=\color{white},   % choose the background color; you must add \usepackage{color} or \usepackage{xcolor}
  basicstyle=\footnotesize,        % the size of the fonts that are used for the code
  breakatwhitespace=false,         % sets if automatic breaks should only happen at whitespace
  breaklines=true,                 % sets automatic line breaking
  captionpos=b,                    % sets the caption-position to bottom
  commentstyle=\color{mygreen},    % comment style
  deletekeywords={...},            % if you want to delete keywords from the given language
  %escapeinside={\%*}{*)},          % if you want to add LaTeX within your code
  %extendedchars=true,              % lets you use non-ASCII characters; for 8-bits encodings only, does not work with UTF-8
  %frame=single,	                % adds a frame around the code
  keepspaces=true,                 % keeps spaces in text, useful for keeping indentation of code (possibly needs columns=flexible)
  keywordstyle=\color{blue},       % keyword style
  language=[ANSI]C,                % the language of the code
  %otherkeywords={*,...},           % if you want to add more keywords to the set
  numbers=left,                    % where to put the line-numbers; possible values are (none, left, right)
  numbersep=5pt,                   % how far the line-numbers are from the code
  numberstyle=\tiny\color{mygray}, % the style that is used for the line-numbers
  rulecolor=\color{black},         % if not set, the frame-color may be changed on line-breaks within not-black text (e.g. comments (green here))
  showspaces=false,                % show spaces everywhere adding particular underscores; it overrides 'showstringspaces'
  showstringspaces=false,          % underline spaces within strings only
  showtabs=false,                  % show tabs within strings adding particular underscores
  stepnumber=1,                    % the step between two line-numbers. If it's 1, each line will be numbered
  stringstyle=\color{mymauve},     % string literal style
  tabsize=2,	                   % sets default tabsize to 2 spaces
  title=\lstname,                  % show the filename of files included with \lstinputlisting; also try caption instead of title
  morecomment=[s]{/*}{*/}
}

En el presente capítulo se describe la arquitectura del sistema, el diseño y la implementación del hardware y del software y las herramientas de desarrollo utilizadas. 


%----------------------------------------------------------------------------------------
%	SECTION 1
%----------------------------------------------------------------------------------------
\section{Arquitectura del sistema}

En esta sección se explica la arquitectura propuesta, junto a los módulos implementados y los protocolos de comunicación utilizados para la conexión de los mismos. También se expone como se piensa lograr la escalabilidad del sistema.

En la figura \ref{fig:tp-final-infra} se muestra el diagrama en bloques del sistema, junto a los módulos, las tecnologías utilizadas y los protocolos de comunicación que los conectan.

\vspace{0.7cm}
\begin{figure}[ht]
	\centering
	\includegraphics[width=1\textwidth]{./Figures/tp-final-infra.png}
	\caption{Diagrama en bloques del sistema implementado.}
	\label{fig:tp-final-infra}
\end{figure}

\pagebreak
\subsection{Módulos del sistema}

El trabajo desarrollado se divide en los siguientes módulos:

\begin{itemize}
\item Módulo de sensado: es el encargado de leer la tarjeta RFID del personal de tercero que quiere ingresar a la planta y enviar la información al módulo de backend para analizar si la persona cumple con los requisitos para acceder o no.
\item Módulo actuador: es el encargado de comandar la cerradura electrónica que permite o evita el ingreso del tercero a la locación industrial. El mismo recibe las órdenes de cómo operar desde el módulo de backend.
\item Módulos de backend y frontend: si bien estos módulos están implementados de manera conjunta en un único servidor de aplicación, cumplen funciones diferentes:

	\begin{itemize}
	\item Módulo de frontend: es el encargado de brindar una interfaz gráfica a los usuarios, para que estos puedan interactuar con la aplicación, recibiendo sus solicitudes y proporcionándoles la información en un formato simple.
	\item Módulo de backend: es quien gestiona todas las solicitudes provenientes del módulo de sensado y del módulo de frontend. Es el encargado de analizar las solicitudes y responder a las mismas. Cuando recibe peticiones del frontend responde al mismo. Cuando recibe peticiones del módulo de sensado, actúa enviándole órdenes o comandos al módulo actuador. También es el encargado de comunicarse con el sistema de gestión de documentación de terceros. Para cumplir con sus funciones utiliza información almacenada en los servidores de base de datos.
	\end{itemize}
\item Sistema de gestión de documentación de terceros: este sistema es externo y no fue parte del desarrollo. El módulo de backend utiliza el mismo para obtener información de los terceros y en base a ésta tomar las decisiones de habilitar o inhabilitar el acceso y generar alertas y tareas de control.
\end{itemize}

\subsection{Protocolos de comunicación entre módulos}
Para la comunicación entre los módulos se utilizaron invocaciones HTTP GET y POST. 
Tomando como referencia el modelo TPC/IP \citep{WEBSITE:modeloTCPIP}, en la tabla \ref{tab:protocolosComunicacionCap3} se muestra el detalle de protocolos empleados en cada capa:


\begin{table}[h]
	\centering
	\caption[Protocolos comunicación]{Protocolos de comunicación empleados por el sistema.}
	\begin{tabular}{p{3.5cm} p{8.5cm} } 	

		\toprule
		\textbf{Capa del modelo} & 
		\textbf{Protocolo}
		\\
		\midrule

Aplicación & HTTP (utilizando los verbos GET y POST)\\ 
Transporte & TCP\\
Internet & IP (IPv4)\\
Acceso al medio & Wi-Fi (802.11n) para la comunicación entre módulos.

Wi-Fi o IEEE 802.3 (Ethernet) para la comunicación entre el usuario y el módulo de frontend.\\
		\bottomrule
		\hline
	\end{tabular}
	\label{tab:protocolosComunicacionCap3}
\end{table}

Para la elección de los protocolos, se tomó en cuenta las tecnologías disponibles en la empresa. Además, al utilizar protocolos abiertos, estándares y extendidos mundialmente, se logró un sistema portable y adaptable.

\subsection{Tecnologías de bases de datos}

El sistema en general y el módulo de backend en particular, se soporta en dos bases de datos:

\begin{itemize}
\item Una base de datos relacional, implementada en PostgreSQL, que es la que contiene todos los objetos necesarios para la aplicación: usuarios, sensores, actuadores, terceros, eventos del sistema, tareas y sub-tareas de control. 
\item Una base de datos no relacional, implementada en MongoDB, que es utilizada por el backend para almacenar la relación entre los eventos de entrada y el conjunto de acciones que se deben tomar en función de dichos eventos. La decisión de utilizar una base no relacional se debe a que cada tipo de evento de entrada genera diferentes tipos de acciones de salida. Por ejemplo, en el caso de un evento de ingreso de un tercero con documentación en regla solo se debe realizar una acción de apertura de cerradura para el módulo actuador. Pero para un evento de ingreso con documentación vencida se deben generar acciones para cerrar la cerradura en el módulo actuador, generar tareas de control para diferentes sectores de planta y enviar un mail a las personas definidas por la gerencia de la empresa. 

\end{itemize}

\subsection{Contenedores docker y escalamiento}

A fin de generar una solución escalable y modular, se utilizaron contenedores docker para implementar el módulo de backend, el módulo de frontend y para levantar las instancias de base de datos, tanto PostgreSQL como MongoDB. Esta decisión permitió:

\begin{itemize}
\item Simplificar el \textit{deploy} de la aplicación: facilitando la configuración del servidor o servidores donde se ejecuta el sistema.
\item Lograr la escalabilidad futura de la solución: al permitir utilizar un orquestador de contenedores como Kubernetes que permite crear o eliminar instancias de cada contenedor dinámicamente en función de diferentes variables, como el consumo de recursos o la cantidad de solicitudes por segundo. Una ventaja adicional es que, si migramos la solución a la nube, al utilizar este esquema de contenedores dinámicos podemos reducir el costo del servicio, dado que estaremos pagando solo por los contenedores que necesitamos en cada instante de tiempo, sin necesidad de tener un número fijo de recursos en todo momento.
\end{itemize}

\pagebreak
\section{Detalle de módulos de hardware}

En esta sección se describe detalladamente la implementación de los dos módulos de hardware desarrollados en el proyecto. El módulo sensor, encargado de la lectura de las tarjetas RFID del personal de tercero, y el módulo actuador, encargado de gestionar la cerradura electrónica para permitir o evitar el ingreso de dicho personal.

\subsection{Módulo sensor}

Es el encargado de leer las tarjetas RFID del tercero y enviar el valor que tiene la misma al módulo de backend.

Cada tarjeta RFID tiene un valor numérico guardado de 4 caracteres de longitud. Las tarjetas permiten definir valores de hasta 16 caracteres, pero dado que la empresa utiliza códigos de 4 caracteres se colocó ese límite para tener uniformidad.

El módulo está compuesto por los siguientes componentes:

\begin{itemize}
\item Un lector de tarjetas RFID RC522. La elección del mismo se debió a su bajo costo, alta disponibilidad en el mercado y su capacidad para leer las tarjetas que tiene la empresa, que operan en la frecuencia de 13,56 MHz.
\item Un \textit{SoC} (System on a chip) ESP32-WRROM-32. La elección del mismo se debió a su bajo costo, alta disponibilidad en el mercado, facilidad de programación y soporte de redes Wi-Fi (normas 802.11 b/g/n). Esto último simplifica la comunicación del módulo con el backend y evita tener que conectarse a la red LAN de la empresa, lo que hubiera requerido hacer una extensión del cableado de la misma.
\item Un conjunto de leds, que permite al usuario conocer el estado del sistema y el estado de sus interacciones con el módulo. Para ello se dispuso un grupo de 3 leds generales de control y otro de 3 leds de respuesta ante las comunicaciones con el backend.

\end{itemize}

En la figura \ref{fig:moduloSensor} se muestra el módulo sensor junto a sus componentes.

\begin{figure}[ht]
	\centering
	\includegraphics[width=1\textwidth]{./Figures/moduloSensor.png}
	\caption{Módulo sensor junto a sus componentes.}
	\label{fig:moduloSensor}
\end{figure}

\subsubsection{Configuraciones y variables del módulo}

Para implementar este módulo, se desarrolló un programa en el entorno Arduino IDE. El mismo lee las tarjetas RFID y se comunica con el backend, enviando los datos requeridos para procesar el intento de ingreso. Dicha comunicación se realiza a través de la red Wi-Fi de la empresa.

\pagebreak
El módulo cuenta con un conjunto de variables a configurar para su correcta operación:

\begin{itemize}
\item WIFI\_SSID: especifica el SSID de la red Wi-Fi de la empresa.
\item WIFI\_PASSWORD: especifica el password de la red Wi-Fi de la empresa.
\item ID\_SENSOR: especifica el ID que tiene el sensor en la base de datos del sistema. Los datos de cada sensor se almacenan en dicha base de datos, la cual incluye su estado, descripción, ubicación y token asociado.
\item tokenlocal: especifica el token de 20 caracteres que tiene asociado el sensor en la base de datos. Con este valor se asegura la autenticación del módulo.
\item servicioAPISensor: contiene la URL del endpoint que expone el backend para recibir los datos de este módulo.
\end{itemize}

\subsubsection{Comunicación con el backend}

El módulo tiene configurada la dirección URL del endpoint que el backend expone para permitir la comunicación entre éstos.

Para enviar los datos solicitados, se realiza un HTTP POST con un objeto JSON que tiene 3 claves:

\begin{itemize}
\item id: contiene el valor ``ID\_SENSOR'' del módulo.
\item token: contiene el valor de ``tokenlocal'' del módulo.
\item valor: contiene el valor leído de la tarjeta RFID, que representa al id del tercero en el sistema.
\end{itemize}

Una vez enviado el HTTP POST, el módulo recibe como respuesta un valor que indica si los datos mandados son correctos o si hubo algún error. Con esta respuesta se determina qué leds deben activarse para dar \textit{feedback} al usuario del estado del proceso.

\subsubsection{Leds del sistema}

El módulo cuenta con un conjunto de leds, que permiten al usuario conocer el estado del mismo y el resultado de sus interacciones con éste.

Cuando el mismo se inicializa hace un chequeo de estos leds, prendiéndolos y apagándolos, uno a uno, durante medio segundo.

En la tabla \ref{tab:combinacionLedsSensor} se muestra el detalle de los leds o combinaciones posibles de leds, junto a la información que brindan al usuario cuando se encienden.

\begin{table}[h]
	\centering
	\caption[Leds módulo sensor ]{Combinación de leds e información para el usuario cuando se prenden.}
	\begin{tabular}{p{4cm} p{8.5cm} } 	

		\toprule
		\textbf{Led/combinación de leds} & 
		\textbf{Información para el usuario}
		\\
		\midrule

PinNoWIFI & Al leer la tarjeta del tercero, si no se cuenta con comunicación Wi-Fi con el backend, el led se prende durante 3 segundos.\\ 
PinTarjNoLeida & Falló la lectura de la tarjeta o la misma no tiene valor asignado. Se debe configurar la tarjeta con el valor correspondiente al personal de tercero.\\
PinTarjLeida & Al acercar la tarjeta al lector, el sistema lee correctamente la misma, junto al valor que tiene almacenado.\\
PinOkSistema & Una vez leída la tarjeta del personal de tercero, el sistema se comunica correctamente con el backend. Se informa al usuario al prender el led durante 2 segundos. \\
PinNoOkSistemaCodigo & Una vez leída la tarjeta del personal de tercero, el sistema se comunica correctamente con el backend, pero el valor de ``ID\_SENSOR'' enviado no se corresponde con ningún módulo sensor configurado en el sistema, o el mismo está inactivo. Se informa al usuario al prender el led durante 2 segundos. \\
PinNoOkSistemaCodigo
+
PinNoOkSistemaInactivo & Una vez leída la tarjeta del personal de tercero, el sistema se comunica correctamente con el backend, pero éste devuelve un error con un código no especificado. Se informa al usuario al prender el ``led PinNoOkSistemaCodigo'' durante 1 segundo seguido del led ``PinNoOkSistemaInactivo'' durante otro segundo.
\\
		\bottomrule
		\hline
	\end{tabular}
	\label{tab:combinacionLedsSensor}
\end{table}


\subsection{Módulo actuador}

Es el encargado de comandar la cerradura. El mismo cuenta con un conjunto de leds que brindan información al personal de tercero del estado del módulo y del estado de su ingreso.

Las órdenes de cómo operar las recibe desde el módulo de backend, para lo cual el actuador expone un \textit{endpoint} HTTP, que recibe un JSON con dichas órdenes.

\pagebreak
El módulo está compuesto por los siguientes componentes:

\begin{itemize}
\item Un \textit{SoC} ESP32-WRROM-32. La elección del mismo se debió a su bajo costo, alta disponibilidad en el mercado, facilidad de programación y soporte de redes Wi-Fi (normas 802.11 b/g/n). Esto último simplifica la comunicación del módulo con el backend y evita tener que conectarse a la red LAN de la empresa, lo que hubiera requerido hacer una extensión del cableado de la misma.
\item Un regulador de tensión, que brinda los niveles de tensión requeridos para energizar el ESP32 y para la activación del Mosfet IRF520. 
\item Cerradura electrónica. La misma se acciona y alimenta desde el Mosfet IRF520.
\item Conversor de niveles lógicos. Se utiliza para convertir la tensión de salida del ESP32 (3.3 V) a la tensión requerida para accionar el Mosfet IFR520 (5V).
\item Mosfet IRF520: permite accionar la cerradura electrónica, brindando el nivel de tensión requerida por la misma (12 V).
\item Un conjunto de leds, que permiten conocer si el actuador está encendido y el estado del ingreso del tercero (habilitado/inhabilitado/error).
\item Fuente de alimentación de 12 V, que es utilizada para alimentar el  Mosfet IRF520 y al regulador de tensión.
\end{itemize}

En la figura \ref{fig:moduloActuador} se muestra el módulo actuador con sus componentes.

\begin{figure}[ht]
	\centering
	\includegraphics[width=1\textwidth]{./Figures/moduloActuador.png}
	\caption{Módulo actuador junto a sus componentes.}
	\label{fig:moduloActuador}
\end{figure}

\subsubsection{Configuraciones y variables del módulo}

Para implementar el actuador, se desarrolló un programa en el entorno Arduino IDE. Este programa expone un \textit{endpoint} HTTP POST, que recibe un JSON con las acciones a realizar. En función de la acción y valor indicados, se acciona la cerradura electrónica y se prenden los leds de control.

En la base de datos del sistema se guarda la información de los actuadores existentes, su ubicación, descripción, estado, dirección IP y token de autenticación.

El actuador cuenta con un conjunto de variables a configurar para su correcta operación:

\begin{itemize}
\item WIFI\_SSID: especifica el SSID de la red Wi-Fi de la empresa.
\item WIFI\_PASSWORD: especifica el password de la red Wi-Fi de la empresa.
\item tokenlocal: especifica el token de 20 caracteres que tiene asociado el actuador en la base de datos. Con este valor se asegura la autenticación del módulo.
\item local\_IP: especifica la dirección IP del mismo. Se utiliza una IP fija, para asegurar que el \textit{endpoint} expuesto siempre pueda ser accedido. Si se utilizara una IP dinámica la dirección podría cambiar y quedaría inaccesible dicho \textit{endpoint}.
\item Subnet: dirección de sub-red de la red Wi-Fi.
\end{itemize}

\subsubsection{Comunicación desde el backend}

El módulo recibe un HTTP POST desde el backend con un objeto JSON que tiene 3 claves:

\begin{itemize}
\item token: contiene el token de autenticación.
\item acción: contiene la acción a realizar. Es un clasificador de acciones posibles.
\item valor: contiene el valor particular para la acción.
\end{itemize}

Al recibir el objeto se controla si el token coincide con el valor de token que se tiene almacenado localmente, y luego se controla si la acción y valor son válidos. En función de la acción y valor, se abre o cierra la cerradura, y prende el led de ingreso ok, de ingreso no ok o de error. Por último, se responde al backend con un código de error o un ok.

\subsubsection{Detalle de respuestas ante solicitudes del backend}

El backend realiza solicitudes al actuador como se explica en la sub-sección anterior. En la tabla \ref{tab:respuestasActuadorBackend} se muestra el detalle las diferentes combinaciones de valores que puede recibir el módulo en las solicitudes y se indica la respuesta brindada al usuario y al backend.


\begin{table}[h]
	\centering
	\caption[Respuestas backend ]{Respuestas posibles del módulo al usuario y al backend ante las solicitudes recibidas.}
	\begin{tabular}{p{4cm} p{4.5cm} p{4.5cm} } 	

		\toprule
		\textbf{Valores recibidos} & 
		\textbf{Respuesta al usuario} &
		\textbf{Respuesta al backend} 
		\\
		\midrule

Acción=``APERTURA''

Valor=``ABRIR''

Token con valor correcto.& Se prende el led verde de manera intermitente durante 4 segundos. Durante ese tiempo la cerradura electrónica se cierra. & Se envía respuesta HTTP con código 200 y mensaje OK. \\
Acción=``APERTURA''

Valor=``CERRAR''

Token con valor correcto. & Se prende el led rojo durante 2 segundos. & Se envía respuesta HTTP con código 200 y mensaje ``Sin token de autenticación.'' \\
Sin token. & Se prende el led amarillo  durante medio segundo y se apaga. & Se envía respuesta HTTP con código 401 y mensaje ``Sin token de autenticación.'' \\
Token con valor incorrecto. & Se prende el led amarillo  durante medio segundo y se apaga. & Se envía respuesta HTTP con código 403 y mensaje ``Token de autenticación incorrecto.'' \\
Acción no especificada o con valor incorrecto. & Se prende el led amarillo de manera intermitente durante 2 segundos. & Se envía respuesta HTTP con código 400 y mensaje ``La acción especificada no es válida.'' \\
Valor no especificado o valor incorrecto. & Se prende el led amarillo de manera intermitente durante 3 segundos. & Se envía respuesta HTTP con código 400 y mensaje ``El valor especificado no es válido''. \\
		\bottomrule
		\hline
	\end{tabular}
	\label{tab:respuestasActuadorBackend}
\end{table}

\pagebreak
\section{Detalle de módulos de software}

En esta sección se describe detalladamente la implementación de los dos módulos de software desarrollados en el proyecto: el de backend, encargado tanto de recibir las solicitudes del módulo sensor y de frontend como de enviar comandos al actuador, y el módulo de frontend, encargado de gestionar las solicitudes del usuario mediante una interfaz gráfica.

\subsection{Módulo de backend}

El mismo está implementado como una aplicación web con Node.JS, utiliza las librerías Express y Socket.io y expone:

\begin{itemize}
\item Una API Rest para el frontend, la cual responde a sus solicitudes e incluye un WebSocket para mostrar alertas online a la Portería.
\item Una API de autenticación, utilizada para la gestión e inicio de sesión de los usuarios.
\item Un \textit{endpoint} para recibir las solicitudes de ingreso del módulo sensor.
\end{itemize}

Para su desarrollo se utilizó el IDE Visual Studio Code. Dentro del mismo se organizaron las carpetas con el código y las configuraciones para el testing automático.

En la figura \ref{fig:backendCarpetas}  se muestra la estructura en carpetas definidas para el backend.

\begin{figure}[ht]
	\centering
	\includegraphics[width=1\textwidth]{./Figures/backendCarpetas.png}
	\caption{Estructura de directorio del backend desarrollado en Visual Studio Code.}
	\label{fig:backendCarpetas}
\end{figure}

En la tabla \ref{tab:carpetasBackend}  se expone detalladamente el contenido y función de cada una de las carpetas y archivos del módulo.

\begin{table}[h]
	\centering
	\caption[Carpetas backend ]{Detalle de archivos y carpetas del módulo de backend.}
	\begin{tabular}{p{2cm} p{11cm}} 	
		\toprule
		\textbf{Archivo/
		Directorio} & 
		\textbf{Descripción} 
		\\
		\midrule
apiDatos & Contiene cada uno de los \textit{endpoints} expuestos al frontend. Se implementó utilizando Express, junto a métodos GET y POST, para servir las consultas de información y las altas y actualizaciones en la base de datos. \\
config & Contiene tanto la configuración del \textit{secret} necesaria para el sub-módulo de autenticación, como la configuración de constantes utilizadas en la comunicación con el sistema de documentación de terceros. \\
controllers & Cuenta con tres controladores:
\begin{itemize}
\item generacionAlertas: es el encargado de comunicarse con el módulo actuador y enviarle los comandos para habilitar o prohibir la solicitud de ingreso del tercero a la planta.
\item generacionMensajes: es el encargado de gestionar el envío de emails a los usuarios requeridos.
\item generacionTareas: es el encargado de generar las tareas y sub-tareas, guardando la información en la base de datos.
\end{itemize} \\
DAL & La DAL (Data Access Layer), es la encargada de abstraer la comunicación con la base de datos, al brindar un conjunto de métodos para acceder y realizar las altas, bajas y modificaciones, sin necesidad de que los componentes que la usan conozcan la implementación subyacente. \\
database & Contiene las configuraciones necesarias para conectarse a la base de datos PostgreSQL. Se utiliza un \textit{pool} de conexiones a fin de mejorar el rendimiento y la escalabilidad del sistema. \\
logger & Es el encargado de gestionar el \textit{logging} de eventos.\\
logs & Almacena los \textit{logs} del sistema. Se guarda un archivo de \textit{log} por día para evitar archivos muy extensos y simplificar la búsqueda de información en los mismos. \\
main

Controller & Es el encargado de gestionar las solicitudes de ingreso del módulo sensor. Se comunica con el sistema de documentación de terceros, determina los tipos de acción a realizar y dispara cada una de ellas, invocando a los controladores de la carpeta controllers.  \\
middleware & Contiene los métodos necesarios para la gestión de la autenticación \\
routes & Contiene cada uno de los \textit{endpoints} expuestos tanto para la API de autenticación (sub-carpeta ``routerAuth'') como para la recepción de datos desde el módulo sensor (sub-carpeta ``routerSensores''). \\
sistDecision & Contiene el sub-módulo que se encarga de determinar las acciones a realizar cuando hay una solicitud de ingreso de un tercero, para lo cual utiliza la base de datos implementada en MongoDB. \\
test & Contiene los archivos necesarios para la ejecución de las pruebas automáticos implementados para la solución. En el capítulo \ref{Chapter4} se explican con mayor detalle las pruebas implementadas y los archivos utilizados. \\
index.js & Contiene la configuración para levantar la aplicación y cada una de las rutas utilizadas por la aplicación. También incluye la configuración de CORS y del WebSocket. \\
		\bottomrule
		\hline
	\end{tabular}
	\label{tab:carpetasBackend}
\end{table}

\clearpage
\subsubsection{API de autenticación}

La API de autenticación se utiliza para segurizar las invocaciones realizadas al backend. Para hacerlo emplea tokens JWT. Además, permite el alta de nuevos usuarios, gestionar el inicio de sesión de los mismos y el cambio y reseteo de passwords.

Para el desarrollo de esta API utilizamos 2 librerías disponibles en node.JS: bcryptjs y jsonwebtoken. La primera permite implementar una función de \textit{hash}, que posibilita guardar encriptado el password de los usuarios. La segunda permite generar el token JWT que es entregado al usuario para que pueda acceder a los diferentes \textit{endpoints}, asegurando su autenticidad. Dicho token tiene una duración de 24 horas.

\subsubsection{Funcionamiento del módulo ante una solicitud del módulo sensor}{\label{sec:subSeccionSolitiudModuloSensor}}   

Cuando el módulo sensor hace una solicitud al backend invoca al \textit{endpoint} de recepción de sensores. El backend, por su parte, recibe la solicitud y realiza los pasos descriptos a continuación: 

\begin{enumerate}
\item Envía el pedido al router ``routerSensores''. 
\item ``routerSensores'' controla el token y valores recibidos. Si el token no es válido o el id de sensor enviado no es correcto o está inactivo, se envía un mensaje de error al origen y se termina la solicitud.
\item Si los datos son correctos, se envían al ``mainController''.
\item El ``mainController'' realiza estas acciones:
	\begin{enumerate}
	\item Se comunica con el sistema de documentación de terceros para determinar si la persona está en condiciones de ingresar. 
	\item Registra el evento de ingreso en la base de datos (no directamente, sino a través de la ``DAL'').
	\item Con los datos obtenidos se comunica con el sub-módulo de decisión (``sistDecision''), el cual le indica las acciones a realizar. 
	\item Para cada una de las acciones indicadas, en función del tipo que sea (de salida, mensaje, tarea), se comunica con los controladores ``generacionAlertas'', ``generacionMensajes'' o ``generacionTareas'', para que éstos las procesen y registren.
	\end{enumerate}
\end{enumerate}

En la figura \ref{fig:DiagramaInteaccion1} se muestra la interrelación entre los componentes del módulo y el flujo de datos ante una solicitud desde el módulo sensor.

\begin{figure}[ht]
	\centering
	\includegraphics[width=1\textwidth]{./Figures/DiagramaInteaccion1.png}
	\caption{Interacción entre los componentes del módulo ante una solicitud del módulo sensor.}
	\label{fig:DiagramaInteaccion1}
\end{figure}

\pagebreak
\subsubsection{Funcionamiento del módulo ante una solicitud del módulo de frontend}

Cuando el módulo de frontend hace una solicitud al backend invoca algunos de los \textit{endpoints} definidos en ``apiDatos''. El backend, por su parte, recibe la solicitud y realiza los pasos descriptos a continuación: 
\begin{enumerate}
\item Envía el pedido a ``apiDatos''. 
\item ``apiDatos'' determina el endpoint solicitado, pasa el control al mismo y éste realiza las siguientes acciones:
	\begin{enumerate}
	\item Controla que la solicitud tenga el token de autenticación y lo valida utilizando las funciones del \textit{middleware} de autenticación.
	\item Si el token no es correcto se rechaza el pedido con un código 403.
	\item Si el token es correcto, opcionalmente y según la necesidad de cada \textit{endpoint}, controla el rol de usuario asociado al token utilizando nuevamente el \textit{middleware} de autenticación.
	\item Si el rol/roles solicitados no son correctos rechaza el pedido con un código 403.
	\item Si los roles son correctos procede con la solicitud. En general cada solicitud controla los datos de entrada y luego se comunica con la base de datos a través de la ``DAL'', ya sea para consultar, agregar o modificar información.
	\end{enumerate}

\end{enumerate}

En la figura \ref{fig:DiagramaInteraccion2} se muestra la interrelación entre los componentes del módulo y el flujo de datos ante una solicitud desde el frontend.

\begin{figure}[ht]
	\centering
	\includegraphics[width=1\textwidth]{./Figures/DiagramaInteraccion2.png}
	\caption{Interacción entre los componentes del módulo ante una solicitud del frontend.}
	\label{fig:DiagramaInteraccion2}
\end{figure}

\subsection{Módulo de frontend}

Este módulo brinda una interfaz gráfica al usuario a través de la cual interactúa con el sistema, ya sea para consultar datos o para registrar acciones. Con el objetivo de cumplir con tales funciones se comunica con el backend a través de una API Rest. Para su desarrollo se empleó Angular y el framework Ionic. Su utilización permitió construir el sistema como una aplicación web responsive con la idea de implementarla a futuro como una \textit{app mobile}. Para la escritura del código fuente apelamos al IDE Visual Studio Code. Dentro del mismo se organizaron las carpetas con el código y cada uno de los diferentes elementos.

\pagebreak
En la figura \ref{fig:frontendCarpetas} se muestra la estructura en carpetas definidas para el frontend.

\begin{figure}[ht]
	\centering
	\includegraphics[width=1\textwidth]{./Figures/frontendCarpetas.png}
	\caption{Estructura de directorio del frontend desarrollado en Visual Studio Code.}
	\label{fig:frontendCarpetas}
\end{figure}


En la tabla \ref{tab:carpetasFrontend}  se expone detalladamente el contenido y función de cada una de las carpetas y archivos del módulo.

\begin{table}[h]
	\centering
	\caption[Carpetas frontend ]{Detalle de archivos y carpetas del módulo de frontend.}
	\begin{tabular}{p{2.3cm} p{10.7cm}} 	
		\toprule
		\textbf{Archivo/
		Directorio} & 
		\textbf{Descripción} 
		\\
		\midrule
cambio-password & Contiene la página que gestiona el cambio de password de los usuarios. \\
components & Contiene tres sub-carpetas con los componentes desarrollados para la solución. Los componentes implementados son:
\begin{itemize}
\item estadística-evento-fecha: muestra la cantidad de ingresos habilitados y rechazados en el día.
\item listar-tareas: muestra una grilla con el listado de tareas y sub-tareas en un rango de fechas y con un estado particular. 
\item tareas-usuario: muestra un listado con cada una de las sub-tareas que tiene un usuario.
\end{itemize} \\
estadísticas-ingreso-mes & Contiene la página que muestra las estadísticas de cantidad de ingresos por mes.\\
estadisticas-tareas-mes & Contiene la página que muestra las estadísticas de cantidad de tareas cerradas por año.\\
gestión-usuarios & Contiene la página que muestra el listado de usuarios del sistema y permite cambiar el estado de los mismos (activo/inactivo) y agregarles o quitarles roles. \\
helpers & Contiene la clase ``authInterceptor'' que permite enviar cada solicitud al backend con el token de autenticación. Para esto intercepta el pedido HTTP y le agrega al encabezado dicho token. \\
home & Contiene la página principal de la aplicación que muestra, según el rol del usuario, las estadísticas de ingreso al sistema o las tareas en curso del mismo.\\
ingresos-hoy & Contiene la página que muestra el listado de ingresos del día con la fecha de cada ingreso y si el usuario fue habilitado o no. \\
listado-tareas & Contiene la página que muestra el listado de tareas en curso. Utiliza el componente ``listar-tareas''. \\
listado-tareas
-historico & Contiene la página que muestra el listado de tareas completas. Utiliza el componente ``listar-tareas''. \\
login & Contiene la página de inicio de sesión para los usuarios. \\
mi-perfil & Contiene la página que muestra el perfil de usuario y sus datos. \\
mis-tareas & Contiene la página que muestra las tareas en curso asignadas al usuario. \\
mis-tareas
-historico & Contiene la página que muestra las tareas cerradas del usuario. \\
model & Contiene las clases que representan a los objetos de negocio del sistema: roles, usuarios, tareas, sub-tareas, sectores. \\
pipes & Contiene la implementación de un \textit{pipe}, que define diferentes colores en función del valor de entrada recibido. Sirve para alertar al usuario de la antigüedad de sus tareas. \\
recupero-password & Contiene la página que permite al usuario recuperar su password. \\
register & Contiene la página que permite dar de alta nuevos usuarios al sistema. \\
services & Contiene los servicios que utilizan las diferentes páginas y componentes para gestionar sus datos y consultas al backend. \\
		\bottomrule
		\hline
	\end{tabular}
	\label{tab:carpetasFrontend}
\end{table}

\clearpage
\subsubsection{Detalle de servicios (``services'') implementados}
En esta sub-sección se detallan los servicios implementados en Angular. Mientras que los componentes y las páginas están enfocados en brindar una interfaz gráfica simple y fácil de utilizar para los usuarios, los servicios se orientan a las tareas de lógica de negocio, lo que incluye comunicarse con el backend y gestionar la autenticación y los datos del usuario.

Los servicios implementados son los siguientes:

\begin{itemize}
\item authService: se comunica con la API de autenticación del backend para gestionar los inicios de sesión, el alta de nuevos usuarios y las funcionalidades de recuperación y cambio de password.
\item camibioMenuService: se encarga del armado del menú de aplicaciones del usuario, en función de su rol. 
\item datosAuxiliaresService: se encarga de comunicarse con el backend para consultar los datos auxiliares del sistema que son de acceso público como, por ejemplo, el listado de sectores de planta para la pantalla de alta de nuevos usuarios.
\item ingresosService: se comunica con el backend para obtener la información de ingresos a planta por rango de fechas.
\item socketService: gestiona el socket utilizado para que la vigilancia y la gerencia puedan visualizar en tiempo real los ingresos a la planta.
\item tareaService: se comunica con el backend para obtener información de las tareas en curso, de las tareas cerradas y para realizar modificaciones en las mismas.
\item tokenStorageService: es el encargado de la gestión del token de autenticación que devuelve el backend al iniciar sesión. Dentro de la gestión se incluye su almacenamiento y recuperación.
\item usuariosService: se comunica con el backend para realizar cambios en el estado de los usuarios y sus roles. Solo es utilizado por el rol administrador. 
\item loginGuardService: permite al módulo de ruteo de la aplicación controlar que el usuario cuente con el rol necesario para acceder a una determinada página. Este servicio se utiliza para habilitar los accesos a las páginas solo al rol de usuario normal.
\item rolAdminGuardService: permite al módulo de ruteo de la aplicación controlar que el usuario cuente con el rol necesario para acceder a una determinada página. Este servicio se utiliza para habilitar los accesos a las páginas solo al rol de usuario administrador.
\item rolAGerenteGuardService: permite al módulo de ruteo de la aplicación controlar que el usuario cuente con el rol necesario para acceder a una determinada página. Este servicio se utiliza para habilitar los accesos a las páginas solo al rol de usuario gerente.
\item rolVigilanciaGuardService: permite al módulo de ruteo de la aplicación controlar que el usuario cuente con el rol necesario para acceder a una determinada página. Este servicio se utiliza para habilitar los accesos a las páginas solo al rol de usuario vigilancia.
\end{itemize}

\subsubsection{Funcionamiento del módulo ante un inicio de sesión}

En este apartado se explica el inicio de sesión de un usuario, en el que se puede ver la interacción con el backend y el guardado del token de autenticación para futuras consultas.
El proceso comienza cuando el usuario ingresa al sistema y visualiza la pantalla de inicio de sesión. Coloca su username y password y hace click en el botón “Loguearse”. Ante el click del usuario, el sistema realiza las siguientes interacciones:

\begin{enumerate}
\item El módulo ``loginModule'' invoca al servicio ``authService'' con los datos de username y password.
\item ``authService'' se comunica con el backend mediante un HTTP POST a la API de autenticación y recibe como respuesta el token asociado al usuario y los datos del mismo (username, password, email, sector y roles asociados). El servicio envía los datos recibidos al módulo ``loginModule''.
\item El módulo al recibir el token y los datos del usuario utiliza el servicio ``tokenStorageService'' para almacenar los valores. Luego, invoca al servicio ``cambioMenuService'' que genera el menú de usuario según sus roles. Por último, invoca al módulo ``homeModule'' que muestra la página de inicio al usuario. 
\end{enumerate}

En la figura \ref{fig:inisioSesionInteraccion} se muestra el diagrama de interacción para el inicio de sesión.

\begin{figure}[ht]
	\centering
	\includegraphics[width=1\textwidth]{./Figures/inisioSesionInteraccion.png}
	\caption{Diagrama de interacción para el inicio de sesión.}
	\label{fig:inisioSesionInteraccion}
\end{figure}

\pagebreak
\subsubsection{Funcionamiento del módulo ante una solicitud de usuario}

En esta sección se explica una interacción típica del usuario en la que se puede ver la relación entre los diferentes elementos del frontend y su comunicación con el backend.

Como pre-requisito, el usuario ya inició sesión en el sistema y desea consultar sus tareas en curso. Para ello hace click en la opción ``Mis Tareas en curso'' del menú de aplicaciones. Ante el click del usuario el sistema realiza las siguientes interacciones:

\begin{enumerate}
\item El módulo de ruteo ``app-routing.module'' determina quién es el encargado de procesar la solicitud del usuario. En nuestro caso es ``mis-tareas-module''. Luego, controla que éste pueda acceder a la página. Para ello consulta con el ``loginGuardService''.
\item Si se determina que se puede acceder a la página, se transfiere el control al módulo ``mis-tareas-module''. Éste invoca al servicio ``tokenStorageService'' para adquirir el token de usuario y su id. Con dicho id llama al servicio ``tareasService'' para obtener las tareas del usuario.
\item El servicio ``tareaService'' genera la solicitud HTTP GET y la envía al backend.
\item El ``authInterceptor'' intercepta la solicitud y le agrega un encabezado con el token de autenticación.
\item El backend procesa la solicitud y devuelve el listado de tareas en curso del usuario.
\item El servicio ``tareaService'' devuelve el listado al módulo ``mis-tareas-module''.
\item El módulo arma con el listado la pantalla necesaria para mostrar la información en un formato simple y la envía al usuario.
\end{enumerate}

En la figura \ref{fig:UsuarioPedidoInteraccion} se muestra el diagrama de interacción para una solicitud de usuario.

\begin{figure}[ht]
	\centering
	\includegraphics[width=1\textwidth]{./Figures/UsuarioPedidoInteraccion.png}
	\caption{Diagrama de interacción para una solicitud de usuario.}
	\label{fig:UsuarioPedidoInteraccion}
\end{figure}

\pagebreak
\subsubsection{Pantalla principal de vigilancia}

Para el rol de vigilancia se implementó un \textit{WebSocket} que posibilita una comunicación en tiempo real con el backend. Esto permite que ante los intentos de ingreso a planta del personal de tercero, el vigilante pueda tener al instante una alerta, ya sea de un acceso exitoso o de un acceso denegado. Para implementar dicho \textit{WebSocket} se utilizó la librería Socket.io, tanto en el backend como en el frontend. 

A continuación, se muestran los pasos que realiza el sistema y los componentes involucrados en la generación de una alerta:
\begin{enumerate}
\item Cuando el usuario de vigilancia inicia sesión en la aplicación, la misma lo redirige a su página principal (módulo ``homeModule''). En ese momento, el frontend utiliza el servicio ``socketService'' para iniciar el WebSocket con el backend y suscribirse a los eventos de tipo ``ingreso''.
\item Posteriormente, cuando un usuario de tercero intenta ingresar a planta, el backend recibe la solicitud de ingreso, la procesa y genera las tareas de control y alertas correspondientes, según lo explicado en la subsección \ref{sec:subSeccionSolitiudModuloSensor}. Dentro de dichas acciones, el sistema emite un evento del tipo ``ingreso'' que contiene el resultado del intento de acceso, los datos del tercero asociado y una descripción con el motivo de la habilitación o denegación.
\item El módulo ``homeModule'' recibe el evento del backend y genera la alerta al vigilante. La misma se presenta en un \textit{Pop up} que se muestra durante 6 segundos e incluye los detalles del acceso. Para ver el detalle de cada tipo de acceso referirse a la sección \ref{sec:PruebasAceptacion}
\end{enumerate}


\section{Interfaz con sistema de documentación}

El sistema de documentación de terceros es un aplicativo que ya está desarrollado en la empresa. El acceso al mismo es a través de un \textit{endpoint} HTTP GET, al cual se le indica el id del usuario de tercero que se quiere consultar, y devuelve un objeto JSON con la información del nombre y apellido de la persona, si el acceso se debe permitir o no (variable con valor OK o NO) y el motivo por el cual se habilita o no a la mismo. Para configurar el acceso a dicho sistema dentro de nuestro desarrollo, bastó con contar con la dirección del \textit{endpoint}. 

Dado que el desarrollo y las pruebas se realizaron en un entorno desconectado de la empresa, se implementó un \textit{mock} para simular el acceso al sistema. Dicho \textit{mock} se desarrolló como una aplicación web con Node.JS y la librería Express, lo que permitió exponer un \textit{endpoint} HTTP GET del mismo modo que lo hace el sistema original. Para el desarrollo se utilizó el IDE Visual Studio Code y se tomaron 4 casos típicos como respuesta:

\begin{enumerate}
\item Usuario activo con documentación en regla.
\item Usuario activo con documentación vencida.
\item Usuario dado de baja (fin de contratación).
\item Usuario no existente (id de usuario nunca dado de alta).
\end{enumerate}

Con estos 4 casos definidos se pudieron probar todas las alternativas y asegurar la respuesta correcta de nuestra aplicación.

% Chapter Template

\chapter{Ensayos y Resultados} % Main chapter title

\label{Chapter4} % Change X to a consecutive number; for referencing this chapter elsewhere, use \ref{ChapterX}

Poner párrafo introductorio.


%----------------------------------------------------------------------------------------
%	SECTION 1
%----------------------------------------------------------------------------------------

\section{Detalle de pruebas realizadas}

Detalle de las pruebas realizadas, herramientas utilizadas para el testing. módulos mockeados.

\section{Pruebas unitarias}

Detalle de pruebas unitarias realizadas sobre los diferentes módulos del sistema.

\subsection{Testing del módulo sensor}

Detalles de pruebas del módulo sensor.

\subsection{Testing del módulo actuador}

Detalle de pruebas del módulo actuador.

\subsection{Testing del módulo de Backend}

Detalle de pruebas del backend del sistema. Esto incluye la API Rest expuesta por el mismo y la API de autenticación.

\section{Pruebas de sistema}

Detalle de prueba integral y de sistema.

\section{Pruebas de aceptación}

Pruebas de aceptación con cliente.

\subsection{Descripción y detalles de prueba de ingreso habilitado}

Detalle de la prueba con ingreso OK de usuario, ejemplo de caso de uso completo del sistema.

\subsection{Descripción y detalles de prueba de ingreso inhabilitado}

Detalle de la prueba con ingreso NO OK de usuario, ejemplo de caso de uso completo del sistema.
 
% Chapter Template

\chapter{Conclusiones} % Main chapter title

\label{Chapter5} % Change X to a consecutive number; for referencing this chapter elsewhere, use \ref{ChapterX}


%----------------------------------------------------------------------------------------

%----------------------------------------------------------------------------------------
%	SECTION 1
%----------------------------------------------------------------------------------------

\section{Resultados obtenidos}

Se logró cumplir el alcance y objetivo del proyecto. En primer lugar, se implementó el sistema de control solicitado que incluyó la puesta en funcionamiento de un módulo sensor, un módulo actuador y una aplicación web para la gestión de tareas de control y de alertas. En segundo lugar, se sentaron las bases para el agregado de futuros casos de uso, para lo cual se hizo un diseño modular. Esto permitirá al sistema el sensado de datos de nuevos procesos de planta, según el requerimiento 1.1, especificado en la sección \ref{sec:Requerimientos}.

Adicionalmente, se incorporó al sistema un módulo para gestionar la autenticación y autorización de los usuarios y el acceso a la API Rest del backend del sistema de modo seguro. Esto posibilitó cumplir con el requerimiento 1.6, que fue agregado al trabajo durante su desarrollo. Este requerimiento permite independizar al sistema desarrollado del sistema de autenticación de la empresa y lograr una mayor portabilidad, lo que le da la posibilidad a futuro de expandir el mismo a otras empresas.

Finalmente, el trabajo cumplió con el requerimiento de lograr una gestión efectiva de los terceros, al implementar un proceso de control estricto de los requisitos de ingreso. Si bien existen otros sistemas de control de ingreso en el mercado, se logró agregar valor mediante la comunicación del sistema en cuestión con el sistema de control de documentación de terceros sumado a los procesos de alerta y gestión implementados. El trabajo realizado habilita una gestión proactiva, rápida y ordenada de los terceros, de forma de actuar inmediatamente ante problemas de ingresos. Las ventajas obtenidas incluyen:

\begin{itemize}
\item Reducir tiempo para la gestión.
\item Evitar atrasos en ingresos por falta de ajustes en la documentación.
\item Evitar problemas legales ante incidentes del personal externo.
\item Evitar el uso de papel y herramientas des-centralizadas para el control y gestión de los terceros por parte de cada sector de la empresa. 
\end{itemize}

Si bien todavía el sistema no se implementó en operativo, con las pruebas realizadas y analizando los ingresos de personal en los últimos dos años, se prevé evitar un 5\% de ingresos incorrectos o con problemas, y reducir los tiempos ante inconvenientes con la documentación en unas 10 horas hombre/mes.

Cabe destacar la importancia de los conocimientos obtenidos a lo largo de la carrera. En primer lugar, fueron muy importantes los aportes de la asignatura de gestión de proyectos. Una buena gestión de proyectos fue fundamental, tanto para lograr una planificación clara que actúe como guía a lo largo de todo el desarrollo, como para minimizar riesgos. En lo particular del trabajo, luego del comienzo del desarrollo se solicitó agregar un nuevo requerimiento al mismo. Dado que se contaba con un diagrama de Gantt \citep{WEBSITE:DiagGantt} detallado y el avance real al momento de la solicitud, se pudo determinar que el nuevo requisito podía cumplirse en tiempo y forma, sin penalizar la fecha de finalización del proyecto. Sin una gestión de proyectos clara, es muy probable que este nuevo requerimiento haya sido rechazado.

Adicionalmente, los conocimientos en desarrollo de aplicaciones multiplataforma permitieron plantear una solución que sea \textit{web responsive} y que a futuro se puede implementar en un entorno mobile con mínimo esfuerzo. También se abre la posibilidad de implementar el sistema en la nube.


%----------------------------------------------------------------------------------------
%	SECTION 2
%----------------------------------------------------------------------------------------
\section{Trabajo futuro}

Para la continuidad y mejora de este trabajo se plantean dos líneas de acción.

Como primera línea de acción, referida al trabajo desarrollado, se incluyen:

\begin{itemize}
\item Realizar mejoras en seguridad:

   \begin{itemize}
      \item Agregar comunicación HTTPS entre los diferentes módulos del sistema. De este modo, se asegura que la información viaje encriptada y se evitan ataques del tipo \textit{man in the middle}. Esto permitirá migrar el sistema a la nube, donde las comunicaciones viajan a través de Internet y no son seguras.
      \item Agregar un segundo factor de autenticación al sistema. Con el objetivo de evitar que ante pérdidas o robos de la tarjeta RFID de ingreso o duplicación de la misma un atacante pueda ingresar a planta, se plantea la posibilidad de agregar un teclado matricial de forma de requerir además de la tarjeta una clave numérica durante el proceso de ingreso.
   \end{itemize}
      
   \item Automatizar tareas de configuración: se analiza implementar una aplicación para poder configurar las acciones del sistema ante las diferentes variantes de entradas (ingreso correcto, ingreso de usuario inactivo, ingreso con documentación vencida). Actualmente esta información se guarda y administra en una base de datos no relacional, por parte del personal de sistemas, que permite definir para cada tipo y valor de entrada un conjunto de acciones de salidas (tareas de control, alertas, emails). Se evalúa desarrollar una aplicación para que el usuario pueda configurar estas salidas y generar diferentes tipos de acción, independizándose del área de sistemas.
   \item Realizar una prueba de implementación en la nube: utilizar la nube de Azure para probar y asegurar el escalamiento de la solución. Esto permitirá incluir nuevas locaciones o plantas industriales al trabajo, ya sea dentro de la empresa actual o para ser implementado en nuevas empresas.
\end{itemize}   

Como segunda línea de acción, en el marco del proyecto integral de gestión de alertas y procesos, el objetivo es incorporar nuevos procesos y casos de uso al sistema. De hecho, ya fue solicitado un primer caso de uso por parte del laboratorio de metrología de la empresa. El mismo implica el control de temperatura y humedad de dicho laboratorio, para asegurar que ambas variables se encuentren dentro de los límites requeridos y generar alertas en caso de desvíos para poder actuar en consecuencia, manual o automáticamente. 
 

%----------------------------------------------------------------------------------------
%	CONTENIDO DE LA MEMORIA  - APÉNDICES
%----------------------------------------------------------------------------------------

\appendix % indicativo para indicarle a LaTeX los siguientes "capítulos" son apéndices

% Incluir los apéndices de la memoria como archivos separadas desde la carpeta Appendices
% Descomentar las líneas a medida que se escriben los apéndices

%\include{Appendices/AppendixA}
%\include{Appendices/AppendixB}
%\include{Appendices/AppendixC}

%----------------------------------------------------------------------------------------
%	BIBLIOGRAPHY
%----------------------------------------------------------------------------------------

\Urlmuskip=0mu plus 1mu\relax
\raggedright
\printbibliography[heading=bibintoc]

%----------------------------------------------------------------------------------------

\end{document}  
